\documentclass[11pt]{article}
\usepackage[utf8]{inputenc}
\usepackage[slovene]{babel}

\usepackage{amsthm}
\usepackage{amsmath, amssymb, amsfonts}
\usepackage{relsize}
\usepackage{mathrsfs}
\usepackage{bbm}
\usepackage{xcolor}
\usepackage{algpseudocode}

\newcommand{\R}{\mathbb{R}}
\newcommand{\N}{\mathbb{N}}
\newcommand{\Z}{\mathbb{Z}}

\renewcommand{\c}{\mathsf{c}}
\newcommand{\set}[1]{\{#1\}}
\newcommand{\oklepaj}[1]{\left(#1\right)}
\newcommand{\1}{\mathbbm{1}}
\newcommand{\ra}{\rightarrow}
\newcommand{\5}{\vspace{0.5cm}}
\newcommand{\Mod}[1]{\ (\mathrm{mod}\ #1)}
\renewcommand{\gcd}[1]{\mathtt{gcd}{(#1)}}

\newcommand{\A}{\mathbf{A}}
\newcommand{\B}{\mathscr{B}}
\newcommand{\C}{\mathscr{C}}
\newcommand{\D}{\mathscr{D}}
\newcommand{\E}{\mathscr{E}}
\newcommand{\K}{\mathscr{K}}

\newcommand{\x}{\mathbf{x}}
\newcommand{\y}{\mathbf{y}}

\theoremstyle{definition}
\newtheorem{definicija}{Definicija}[section]

\theoremstyle{definition}
\newtheorem{trditev}{Trditev}[section]

\theoremstyle{definition}
\newtheorem{izrek}{Izrek}[section]

\theoremstyle{definition}
\newtheorem{metoda}{Metoda}[section]

\newtheorem*{posledica}{Posledica}
\newtheorem*{opomba}{Opomba}
\newtheorem*{komentar}{Komentar}
\newtheorem{lema}{Lema}
\newtheorem*{dokaz}{Dokaz}
\newtheorem*{posplošitev}{Posplošitev}
\newtheorem*{dogovor}{Dogovor}
\newtheorem*{sklep}{Sklep}
\newtheorem*{oznake}{Oznake}
\newtheorem{sifra}{Šifra}
\newtheorem{algoritem}{Algoritem}

\title{Teorija kodiranja in kriptografija - definicije, trditve in izreki}
\author{Oskar Vavtar \\
po predavanjih profesorice Arjane Žitnik}
\date{2021/22}

\begin{document}
\maketitle
\pagebreak
\tableofcontents
\pagebreak

% #################################################################################################

\section{Klasične šifre}
\5

\definicija{\textit{Kriptosistem} je peterka $(\B,\C,\K,\E,\D)$, kjer so
\begin{itemize}
	\item $\B$ končna množica besedil,
	\item $\C$ končna množica kriptogramov,
	\item $\K$ končna množica ključev,
	\item $\E = \set{E_k: \B \ra \C \mid k \in \K}$ družina kodirnih funkcij,
	\item $\D = \set{D_k: \C \ra \B \mid k \in \K}$ družina dekodirnih funkcij,
\end{itemize}

tako, da velja 
$$\forall e \in \K, \exists d \in \K: ~D_d(E_e(x)) ~=~ x, \quad \forall x \in \B.$$}
\5

\trditev{Vsaka kodirna funkcija $E_k \in \E$ je injektivna.}
\5

\sifra[Cezarjeva/s pomikom]{a, b, c,\ldots, ž označimo z 0, 1, 2,\ldots, 24.
\begin{itemize}
	\item $\B = \C = \K = \set{0,1,2,\ldots,24} = \Z_{25}$
	\item kodiranje: $E_k(x) \equiv x + k \Mod{25}$
	\item dekodiranja: $D_k(y) \equiv y - k \Mod{25}$
\end{itemize}
Računamo torej v grupi $(\Z_{25},+)$.}
\5

\algoritem[Izčrpno iskanje ključev]{~
\begin{itemize}
	\item Podatki: $x \in \B, y \in \C$ za kriptosistem $(\B,\C,\K,\E,\D)$
	\item Iščemo: $k \in \K$, za katerega je $E_k(x) = y$
\end{itemize}
\begin{algorithmic}
\For{$k \in \K$}
	\If{$E_k(x) = y$}
		\Return $k$
	\EndIf
\EndFor
\end{algorithmic}
\noindent Deluje za majhne $|\K|$. Kriptosistem je \textit{razbit}, če lahko ključ najdemo ``dosti hitreje'' kot s preverjanjem vseh ključev.}
\5

\subsection*{Razširjen evklidov algoritem}
\5

\izrek[Osnovni izrek o deljenju]{Naj bosta $a \in \Z$ in $b \in \N$. Potem obstajata enolično določeni $q,r \in \Z$, $0 \leq r \leq b$, da velja
$$a ~=~ q \cdot b + r.$$}
\5

\algoritem[Evklidov algoritem]{~
\begin{itemize}
	\item Vhod: $a,b \in \N$
	\item Izhod: $\gcd{a,b}$
\end{itemize} 
\begin{algorithmic}
\State $r_0 = a$
\State $r_1 = b$
\While{$r_{n+1} \neq 0$}
	\State $q_i = r_{i-1}/r_i$
	\Comment{``grdo'' deljenje brez ostanka}
	\State $r_{i+1} = r_{i-1} - q_i \cdot r_i$
\EndWhile \\
\Return $r_n$
\end{algorithmic}}
\5

\algoritem[Razširjeni Evklidov algoritem]{~
\begin{itemize}
	\item Vhod: $a,b \in \N$
	\item Izhod: $(r,s,t)$; $r = \gcd{a,b}$, $s \cdot a + t \cdot b = \gcd{a,b}$
\end{itemize} 
\begin{algorithmic}
\State $r_0 = a$, $s_0 = 1$, $t_0 = 0$
\State $r_1 = b$, $s_1 = 0$, $t_1 = 1$
\While{$r_{n+1} \neq 0$}
	\State $q_i = r_{i-1}/r_i$
	\Comment{``grdo'' deljenje brez ostanka}
	\State $r_{i+1} = r_{i-1} - q_i \cdot r_i$
	\State $s_{i+1} = s_{i-1} - q_i \cdot s_i$
	\State $t_{i+1} = t_{i-1} - q_i \cdot t_i$
\EndWhile \\
\Return $(r_n,s_n,t_n)$
\end{algorithmic}}
\5

\trditev{$$s_n \cdot a + t_n \cdot b ~=~ \gcd{a,b}$$}
\5

\definicija{\begin{align*}
\Z_n^* ~&=~ \set{i \in \set{1,\ldots,n-1} \mid i \perp n} ~=~ \set{i \in \set{1,\ldots,n-1} \mid \gcd{i,n}=1}\\
|Z_n^*| ~&=~ \varphi(n)
\end{align*}
Za $n = p_1^{\alpha_1}\cdot\ldots\cdot p_k^{\alpha_k}$ je
$$\varphi(n) ~=~ (p_1-1)p_1^{\alpha_1 - 1} \cdot \ldots \cdot (p_k-1)p_k^{\alpha_k-1}.$$ }
\5

\sifra[Afina]{~
\begin{itemize}
	\item $\B = \C = \Z_{25}$, $\K = \Z_{25}^* \times \Z_{25}$
	\item ključ: $(a,b) \in \K$
	\item kodiranje: $E_{(a,b)}(x) = ax + b \Mod{25}$
	\item dekodiranje: $D_{(a,b)}(y) = a^{-1}(y-b) \Mod{25}$
\end{itemize}
$$D_{(a,b)}(E_{(a,b)}(x)) ~=~ a^{-1}\oklepaj{(ax+b)-b} ~=~ x$$}
\5

\sifra[Hillova]{~
\begin{itemize}
	\item $\B = \C = \Z_{25}^n$, $\K = \set{\A \in \Z_{25}^{n \times n} \mid \det(\A) \in \Z_{25}^*}$
	\item ključ: $\A \in \K$
	\item kodiranje: $E_\A(\x) = \A\x \Mod{25}$
	\item dekodiranje: $D_\A(\y) = \A^{-1}\y \Mod{25}$
\end{itemize}
$$D_\A(E_\A(\x)) ~=~ \A^{-1}\A\x ~=~ \x$$}
\5



% #################################################################################################


\end{document}